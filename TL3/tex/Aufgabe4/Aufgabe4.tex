\section*{Aufgabe 4}
\subsection*{a)}

\subsection*{b)}

\begin{enumerate}
\item Ein Alternatives Verfahren zum Soundex in der deutschen Sprache ist die Kölner Phonetik.\footnote{Quelle: \url{http://uni-koeln.de/phil-fak/phonetik/Lehre/MA-Arbeiten/Martin_Wilz.pdf}}
\item Einige Unterschiede zum Soundex sind:
	\begin{enumerate}
		\item Es wird auf den Wort Anfang geachtet. Der erste Buchstabe wird auch kodiert. 
		\item Die Buchstaben werden, nicht wie beim Sondex, in Großbuchstaben sondern in Kleinbuchstaben umgewandelt, da es im Deutschen kein großes scharfes s (ß) gibt.
		\item Umlaute erhalten den selben wert wie Vokale.
		\item Es wird auf folgende Buchstaben geachtet, somit kann ein Buchstube je nach Kontext einen anderen Code erhalten. (CH => 4; C=>8)
		\item Die Länge des Codes ist nicht auf 4 Zeichen beschränkt, da deutsche Wörter in der Regel länger sind als englische. 
	\end{enumerate}
\item Ablauf:
	\begin{enumerate}
	\item Buchstabenweise Kodierung von links nach rechts entsprechend der Umwandlungstabelle. 
	\item Entfernen aller mehrfach nebeneinander vorkommenden Ziffern.
	\item Entfernen aller Codes 0 außer am Anfang.
	\end{enumerate}

\end{enumerate}

