\section*{Aufgabe 4}
\begin{itemize}
\item \textit{t}-Test\\
Teststatistik: $t = \frac{\overline{B - A}}{\sigma_{B-A}} \cdot \sqrt{N}$\\
$\overline{B-A} = 4,5 $; $\sigma = 4,63$; $t= 2,75$; $p = 0,006$
\item Wilcoxon-Vorzeichen-Rang-Test\\
Teststatistik: $w = \sum_{i=1}^N R_i$\\
\begin{center}
\begin{tabular}{|c|c|c|c|c|c|c|}
\hline 
Differenz & 2 & 5 & 7 & 7 & 8 & 11 \\ 
\hline 
Vorzeichen & - & + & + & + & + & + \\ 
\hline 
Rang & 1 & 2 & 3,5 & 3,5 & 5 & 6 \\ 
\hline 
Vorzeichenbehafteter Rang & -1 & +2 & +3,5 & +3,5 & 5 & 6 \\ 
\hline 
\end{tabular}
\end{center}
$w = 19$ ist die Summe der vorzeichenbehafteten Rängen
\item Vorzeichentest\\
Die Teststatistik ist 5 (5 mal ist System B besser als System A)\\
\end{itemize}