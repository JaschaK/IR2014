\newpage
\section*{Aufgabe 9}

Beim \glqq TREC Federated Web Search track\grqq  handelt es sich um einen Workshop im Rahmen einer Konferenz, mit dem Forschung an sogenannten Verbundsuchmaschinen gefördert werden sollen. Eine Verbundsuchmaschine tut das was ihr Name schon sagt. Sie stellt eine Suchanfrage bei mehreren Suchmaschinen und bündelt dann die Ergebnisse auf einer Seite. Das Ziel des Projektes ist die Bewertung von Ansätzen für Verbundsuchmaschinen. Hierzu werden die Suchergebnisse von existierenden Suchmaschinen kombiniert um eine möglichst realistische, groß angelegte Umgebung zu schaffen.\\
\ \\
Im Rahmen des \glqq TREC Federated Web Search track\grqq\ gibt es nun drei Aufgaben, sogenannte \glqq Tasks\grqq\ (T1-T3), die in den blauen Kästchen dargestellt werden. Teilnehmer des Projekts können sowohl an T1 \glqq Vertical Selection\grqq\ oder T2 \glqq Resource Selection\grqq\ oder auch an beiden dieser Aufgaben teilnehmen. Als Voraussetzung für die Aufgabe T3 \glqq Results Merging\grqq\ besteht allerdings die Teilnahme an einem \glqq Durchlauf\grqq\ der Aufgabe T2.\\
\ \\
Bei Aufgabe T1 \glqq Vertical Selection\grqq\ geht es darum, dass viele Suchanfragen bestimmten Themenbereichen, Medientypen oder Genres zugeordnet werden können, diese nennt man \glqq Verticals\grqq . Die Teilnehmer sollen zu vorgegeben Suchanfragen aus einer Sammlung von \glqq Verticals\grqq\ die passendsten heraussuchen und zuordnen. Das Ziel ist, dass nicht für jede Suchanfrage alle Quellen durchsucht werden müssen, sondern man sich z.B. bei der Anfrage \glqq Blumen\grqq auf Bilddateien und Enzyklopedia-Einträge beschränkt. Als Ergebnis wird von den Teilnehmern eine sortierte Liste der \glqq Verticals\grqq\ erwartet. Die Güte wird mittels F-Measure bewertet.\\
\ \\
Aufgabe T2 \glqq Resource Selection\grqq\ behandelt die Auswahl von Quellen in denen die Verbundsuchmaschine suchen soll. Aus praktischen Gründen soll hier nicht immer auf alle verfügbaren Quellen zurückgegriffen werden. Hier soll der Output der Teilnehmer eine Liste mit den sortierten Ressourcen sein, die mittels nDCG@20 bewertet wird.\\
\ \\
Aufgabe T3 \glqq Results Merging\grqq\ schließt das ganze dann ab damit, dass aus den vorher in T2 gewählten Quellen die Suchergebnisse in einer einzelnen sortierten Liste zusammengeführt werden. Für jede Anfrage sollen hierbei die Ergebnisse aus den besten 20 Quellen genutzt werden. Als Ergebnis wird eine sortierte Liste der Snippet IDs erwartet, die mittels nDCG@20 und IA-nDCG@20 bewertet wird. Es können von den Teilnehmern höchstens 7 Durchläufe eingereicht werden, wobei mindestens einer davon auf der Basis-Quellen-Auswahl (von TREC) ausgeführt werden muss.