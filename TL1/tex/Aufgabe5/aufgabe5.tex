\section*{Aufgabe 5}
\begin{center}
\begin{tabular}{|l|c|c|c|c|}
\hline 
Anfrage & System A & System B & A $ - $ B & Vorzeichen\\ 
\hline 
1 & 0.4 & 0.2 & 0.2 & $+$\\ 
\hline 
2 & 0.8 & 0.8 & 0 & \\ 
\hline 
3 & 0.8 & 0.2 & 0.6 & $+$\\ 
\hline 
4 & 0.8 & 0.2 & 0.6 & $+$\\ 
\hline 
5 & 0.6 & 0.2 & 0.4 & $+$\\ 
\hline 
6 & 0.6 & 0.4 & 0.2 & $+$\\ 
\hline 
7 & 0.4 & 0.2 & 0.2 & $+$\\ 
\hline 
8 & 0.4 & 0 & 0.4 & $+$\\ 
\hline 
9 & 0.6 & 0.6 & 0 & \\ 
\hline 
10 & 0.4 & 0.2 & 0.2 & $+$\\ 
\hline 
11 & 0.6 & 0.2 & 0.4 & $+$\\ 
\hline 
12 & 0.4 & 0.2 & 0.2 & $+$\\ 
\hline 
13 & 0.6 & 0 & 0.6 & $+$\\ 
\hline 
14 & 0.8 & 0.4 & 0.4 & $+$\\ 
\hline 
15 & 0.2 & 0.4 & $ - $ 0.2 & $-$\\ 
\hline 
16 & 0.6 & 0.4 & 0.2 & $+$\\ 
\hline 
17 & 0.6 & 0 & 0.6 & $+$\\ 
\hline 
18 & 0.4 & 0.2 & 0.2 & $+$\\ 
\hline 
19 & 0.4 & 0.6 & $ - $ 0.2 & $-$\\ 
\hline 
20 & 0.6 & 0.2 & 0.4 & $+$\\ 
\hline 
\end{tabular}
\end{center}
\vspace{1cm}
Vorgehen:\\
Zuerst wird das Effektivitätsmaß P$@$5 für jede Anfrage und beide Systeme berechnet. Dies geschieht einfach in dem man die Anzahl an relevanten Dokumenten durch die Gesamtzahl an Dokumenten teilt. Da die Hypothese ist, dass System A besser sei als System B, wird der Vorzeichentest "zu Gute" von A angewendet. Es wird also geprüft wann P$@$5 bei A größer ist als bei B, also positiv und wann es kleiner ist, also negativ. 
Die Teststatistik für diesen Test beträgt 16.\\
Der $P$-Wert liegt bei 0,0059. (k $\ge$ 16 in der Tabelle der Binomialverteilung)\\
Interpretation: In 16 von 20 Fällen ist System A besser als System B, wobei eine Irrtumswahrscheinlichkeit von ca. 0,6 \% anzunehmen ist.\\
Das Problem beim Effektivitätsmaß P$@$R ist, dass die Verteilung innerhalb der Ränge nicht berücksichtigt wird.\\Bsp: Bei P$A$5: Eine Anfrage mit einem relevantem Dokument an Rang 1 und sonst nicht relevanten Dokumenten ist genauso gut wie eine Anfrage mit einem relevantem Dokument an Rang 5 und sonst nicht relevanten Dokumenten.\\